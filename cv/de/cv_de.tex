\documentclass[12pt,a4paper,sans]{moderncv}

\moderncvstyle{banking}
\moderncvcolor{blue}

\usepackage[utf8]{inputenc}
\usepackage[ngerman]{babel}

\usepackage[scale=0.75]{geometry}

\name{Philipp}{Waack}

\email{waackphilipp@gmail.com} 
\homepage{waackph.github.io}
\quote{Ich arbeite als Full Stack Developer und Data Analyst. Ich bin ein Team Player und eine neugierige und aufgeschlossene Person.}


%----------------------------------------------------------------------------------
%            content
%----------------------------------------------------------------------------------
\begin{document}



% #################################
% Lebenslauf......................

\makecvtitle

\section{Ausbildung}
% arguments 3 to 6 can be left empty
\cventry{2016--2019}{Master of Science; Informatik}{Humboldt Universität zu Berlin}{Note: 1,9}{}{
Masterarbeit: \emph{Empirical study about the influence of social dimensions on the SCHUFA-Score}
}
\cventry{2012--2017}{Bachelor of Science; Medieninformatik}{Technische Universität Dresden}{Note: 2,6}{}{
Bachelorarbeit: \emph{Benutzbarkeitskonzept für VPN-basierte Anonymisierung mobiler Geräte}
}
\cventry{2002--2011}{Abitur; Allgemeine Hochschulreife}{Gymnasium Willhöden, \mbox{Stadtteilschule} Blankenese Framstraße}{Note: 2,5}{Hamburg}{}

\section{Berufserfahrung}
\cventry{seit 05/23}{Referent}{VDI/VDE-IT}{Berlin}{}{
Arbeit in der Abteilung Forschung und Entwicklung für IT Lösungen und Data Science Fragestellungen.
}
\cventry{12/19--03/23}{Data Governance Expert}{acs-plus}{Berlin}{}{
Arbeit an der Schnittstelle zwischen Data Science, Data Engineering und dem Themenkomplex der Data Governance.
}
\cventry{10/17--10/19}{Studentische Hilfskraft}{acs-plus}{Berlin}{Data Scientist}{
Unterstützung und eigenverantwortliche Durchführung von Datenanalysen und -modellen sowie Umsetzung von Anwendungen zur Darstellung von \mbox{Analyseergebnissen}.
}
\cventry{04/16--09/16}{Studentische Hilfskraft}{Zentrum für Informationsdienste und Hochleistungsrechnen (ZIH)}{Dresden}{Redakteur}{
Erstellung redaktioneller Inhalte sowie technischer Anleitungen für das Content Management System des ZIH an der Technischen Universität Dresden.
}

\newpage

\section{Fähigkeiten und Erfahrungen}

\cvline{Programmiersprachen}{\hfill \break In mehreren Game Development Projekten habe ich Erfahrungen mit \underline{C\#} gesammelt. Ich habe Spiele mit Unity, MonoGame and einfachem C\# entwickelt. In Studienprojekten habe ich zudem Erfahrungen mit \underline{Java} sammeln können - insbesondere bei der Implementierung eines Frontends für einen Anonymisierungsdienst als Java basierten Android Applikation für meine Bachelorarbeit. Ich habe zudem Erfahrungen mit \underline{C} gesammelt im Rahmen eines Studienprojekts. Dort habe ich ein Word Embedding Space Model für ein Document Ranking in C portiert.}

\cvline{Scriptingsprachen}{\hfill \break Ich programmiere bereits seit mehreren Jahren privat und beruflich vorwiegend in \underline{Python} und \underline{JavaScript}. Mit diesen Sprachen habe ich mehrere professionelle datengetriebene Web Applikationen entwickelt. Ich verwende überwiegend \underline{Python} und \underline{R} für die Analyse von Kundendaten und in privaten Projekten. Bei der Arbeit und im privaten Kontext benutze ich \underline{Bash scripting} um bestimmte Aufgaben, wie das Erstellen von Backups, zu automatisieren. In meiner ehrenamtlichen Tätigkeit für das Projekt afeefa.de habe ich zudem etwas Erfahrung in der Web Entwickung mit \underline{Ruby} und dem Framework \underline{Ruby on Rails} gesammelt.}

\cvline{Web Technologien \& Frameworks}{\hfill \break Um Web Applikationen zu implementieren habe ich hauptsächlich mit Python \underline{Flask} und der Template Engine \underline{Jinja2} verwendet oder mit Flask REST APIs entwickelt, um beispielsweise mit einem \underline{React} frontend zu kommunizieren. In manchen Projekten habe ich \underline{Express.js} als Backend genutzt. Ich habe zudem Erfahrungen mit \underline{HTML}, \underline{CSS}, \underline{JavaScript} und  für dynamische User Interfaces habe ich hauptsächlihc \underline{React} und \underline{Bootstrap} als Frontend Frameworks verwendet.}

\cvline{Datenanalyse}{\hfill \break In Datenanalyse Projekten habe ich \underline{R} und verschiedene \underline{Python} Bibliotheken verwendet, um Daten zu analysisieren, visualisieren und Modelle zu entwickeln. Für die statistische Datenanalyse und Data Cleaning habe ich vorwiegend mit \underline{pandas} und \underline{SciPy} gearbeitet. Um Daten für Analysen zu visualisieren nutze ich \underline{matplotlib} und \underline{seaborn}. Für interaktive, datengetriebene Web Applikationen nutze ich \underline{D3.js} und \underline{Plotly}. Um statistische Modelle und Machine Learning Modelle zu trainieren nutze ich \underline{sklearn}. Aber ich habe auch Erfahrungen mit \underline{TensorFlow} und \underline{GPflow} während meiner Masterarbeit gesammelt.}

\cvline{Datenbank Technologie}{\hfill \break In den meisten Web Development Projekten und Datenanalyse Projekten habe ich mit Relationalen \underline{SQL Datenbanken} gearbeitet. Ich habe Erfahrungen mit den DBMS MariaDB, PostreSQL und SQL Server. In manchen Projekten habe ich auch mit NoSQL Databanken gearbeitet. Ich habe Erfahrungen mit \underline{MongoDB} gesammelt und in einem Projekt mit der Graph Databank \underline{Neo4j} gearbeitet.}

\cvline{Server Technologie}{\hfill \break Mit \underline{Apache HTTP server} und \underline{Nginx} habe ich verschiedene Web Applikationen für Testing und Deployment aufgesetzt.}

\cvline{Versionierung und Deployment}{\hfill \break Für Code Versionierung und die Kollaboration in agilen Teams habe ich mit \underline{Git} in GitLab and GitHub gearbeitet. Um Testing und Deployment Pipelines aufzubauen arbeite ich abhängig vom Projekt mit \underline{GitLab CI/CD} oder \underline{GitHub Actions}. Projekte habe ich mit \underline{Docker} als containerisierte Applikationen deployt. Für die Programmierung nutze ich überwiegend den Code Editor \underline{Visual Studio Code}.}

\cvline{Game Engines}{\hfill \break Ich habe verschiedene Game Prototypen mit dem Framework \underline{MonoGame} und der Game Engine \underline{Unity} entwickelt.}

\cvline{Betriebssysteme}{\hfill \break Ich arbeite überwiegend mit  \underline{Ubuntu} und \underline{Windows}. Ich habe aber auch Erfahrungen mit \underline{macOS} gesammelt.}

\cvline{Technologien zur Konzeption \& Kollaboration} {\hfill \break Um Frontend Entwürfe und Mockups zu erstellen nutze ich \underline{Figma}. Für die Zusammenarbeit mit anderen sowie konzeptuelle Arbeiten und die Erstellung von Datenbank Modellen und Prozessen nutze ich die kollaborative Plattform \underline{Miro}.}

\cvline{Bildbearbeitung}{\hfill \break Ich habe grundlegende Kenntnisse in \underline{Gimp}, \underline{Inkscape} und \underline{Scribus}. Für 2D Videospielprojekte nutze ich \underline{Aseprite}, um Sprites and Animationen zu erstellen.}

\cvline{Office Anwendungen}{\hfill \break Ich arbeite mit Open Source \underline{Office} Programmen, \underline{PowerPoint} and \LaTeX.}

\section{Sprachen}
\cvitemwithcomment{Deutsch}{Muttersprache}{}
\cvitemwithcomment{Englisch}{Fließend}{}
\cvitemwithcomment{Spanisch}{Grundkenntnisse}{}

\section{Praktika und Nebenjobs}
\cventry{03/14--06/14}{Praktikum}{surfcamplaspalmas.com}{Las Palmas de Gran Canaria}{Web Content Manager, SEO-Optimization}{
Schwerpunkt war die Erstellung von Blogeinträgen rund um das Thema Surfen, die für die Google Suchmaschine optimiert sind.
}
\cventry{03/12--09/12}{Diverse Aushilfsjobs}{Edeka, Schachtrupp, Störmer Murphy and Partners}{Hamburg}{Aushilfe}{}
\cventry{01/12--02/12}{Praktikum}{Hamburger Abendblatt}{Hamburg}{Journalist}{
Praktikum mit Schwerpunkt auf selbstständige, lokaljournalistische Arbeiten im Hamburger Bezirk Bergedorf für die Internetpräsenz der Tageszeitung Hamburger Abendblatt.
}
\cventry{11/11--01/12}{Praktikum}{Elbe Wochenblatt}{Hamburg}{Redakteur}{
Recherche und Erstellung von Meldungen, kurzen Berichten und Artikeln für eine Hamburger Lokalzeitung.
}
\cventry{08/11--10/11}{Praktikum}{Infected Postproduction}{Hamburg}{Mediengestalter für Bild und Ton}{
Schwerpunkt war der Schnitt und die Nachberarbeitung von Werbefilmen.
}

\section{Ehrenamtliche Mitarbeit und Weiterbildungen}
\cventry{09/19--10/19}{Seminar}{Journalistisches Arbeiten}{Berlin}{}{
Seminar mit Schwerpunkt auf Methoden und Regeln zum journalistischen Schreiben. Recherche und Erstellung einer eigenen Reportage.
}
\cventry{09/17--10/17}{Seminar}{Storytelling in Werbung, Journalismus und Politik}{Berlin}{}{
Seminar mit Fokus auf unterschiedliche Erzähltechniken und dramaturgische Grundlagen (etwa Figurenentwicklung und Heldenreise). Entwicklung und Präsentation einer eigenen Geschichte.
}
\cventry{02/17--02/19}{Ehrenamt}{Kunstprojekt Web-Residency: x-temporary.org}{Berlin}{Webdesign}{
KünstlerInnen Residenz im digitalen Raum. Aufgabe waren die Überarbeitung der Webseite und die Unterstützung der KünstlerInnen in technischen Belangen.
}
\cventry{04/15--09/16}{Ehrenamt}{Studentenzeitung ad rem}{Dresden}{Journalist}{
Recherche und Erstellung journalistischer Artikel über Technik- und Digitalthemen, Hochschulpolitik sowie Veranstaltungen. Leitung für das Ressort Technik.
}
\cventry{08/15--04/16}{Ehrenamt}{Web-Applikation afeefa.de}{Dresden}{Webdesigner, Ruby-on-Rails-Entwickler}{
Mitarbeit an einer digitalen Plattform zu Angebot und Teilnahme für geflüchtete Menschen.
}


\end{document}
%% end of file `template.tex'.
