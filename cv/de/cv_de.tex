\documentclass[12pt,a4paper,sans]{moderncv}

\moderncvstyle{classic}
\moderncvcolor{blue}

\usepackage[utf8]{inputenc}
\usepackage[ngerman]{babel}

\usepackage[scale=0.75]{geometry}

\name{Philipp}{Waack}

\email{waackphilipp@gmail.com} 
%\quote{}


%----------------------------------------------------------------------------------
%            content
%----------------------------------------------------------------------------------
\begin{document}



% #################################
% Lebenslauf......................

\makecvtitle

\section{Ausbildung}
% arguments 3 to 6 can be left empty
\cventry{2016--2019}{Master of Science; Informatik}{Humboldt Universität zu Berlin}{Note: 1,9}{}{
Masterarbeit: \emph{Empirical study about the influence of social dimensions on the SCHUFA-Score}
}
\cventry{2012--2017}{Bachelor of Science; Medieninformatik}{Technische Universität Dresden}{Note: 2,6}{}{
Bachelorarbeit: \emph{Benutzbarkeitskonzept für VPN-basierte Anonymisierung mobiler Geräte}
}
\cventry{2002--2011}{Abitur; Allgemeine Hochschulreife}{Gymnasium Willhöden, Stadtteilschule Blankenese Framstraße}{Hamburg}{Note: 2,5}{}

\section{Berufserfahrung}
\cventry{12/19--05/21}{Data Governance Expert}{acs-plus}{Berlin}{}{
Arbeit an der Schnittstelle zwischen Data Science, Data Engineering und dem Themenkomplex der Data Governance.
}
\cventry{10/17--10/19}{Studentische Hilfskraft}{acs-plus}{Berlin}{Data Scientist}{
Unterstützung und eigenverantwortliche Durchführung von Datenanalysen und -modellen sowie Umsetzung von Anwendungen zur Darstellung von Analyseergebnissen.
}
\cventry{04/16--09/16}{Studentische Hilfskraft}{Zentrum für Informationsdienste und Hochleistungsrechnen (ZIH) an der Technischen Universität Dresden}{Dresden}{Redakteur}{
Erstellung redaktioneller Inhalte sowie technischer Anleitungen für das Content Management System des ZIH an der Technischen Universität Dresden.
}

\newpage

\section{Praktika und Nebenjobs}
\cventry{03/14--06/14}{Praktikum}{surfcamplaspalmas.com}{Las Palmas de Gran Canaria}{Web Content Manager, SEO-Optimization}{
Schwerpunkt war die Erstellung von Blogeinträgen rund um das Thema Surfen, die für die Google Suchmaschine optimiert sind.
}
\cventry{03/12--09/12}{Diverse Aushilfsjobs}{Edeka, Firma Schachtrupp, Architektenbüro Störmer Murphy and Partners}{Hamburg}{Aushilfe}{}
\cventry{01/12--02/12}{Praktikum}{Hamburger Abendblatt}{Hamburg}{Journalist}{
Praktikum mit Schwerpunkt auf selbstständige, lokaljournalistische Arbeiten im Hamburger Bezirk Bergedorf für die Internetpräsenz der Tageszeitung Hamburger Abendblatt.
}
\cventry{11/11--01/12}{Praktikum}{Elbe Wochenblatt}{Hamburg}{Redakteur}{
Recherche und Erstellung von Meldungen, kurzen Berichten und Artikeln für eine Hamburger Lokalzeitung.
}
\cventry{08/11--10/11}{Praktikum}{Infected Postproduction}{Hamburg}{Mediengestalter für Bild und Ton}{
Schwerpunkt war der Schnitt und die Nachberarbeitung von Werbefilmen.
}

\section{Ehrenamtliche Mitarbeit und Weiterbildungen}
\cventry{09/19--10/19}{Seminar}{Journalistisches Arbeiten}{Berlin}{}{
Seminar mit Schwerpunkt auf Methoden und Regeln zum journalistischen Schreiben. Recherche und Erstellung einer eigenen Reportage.
}
\cventry{09/17--10/17}{Seminar}{Storytelling in Werbung, Journalismus und Politik}{Berlin}{}{
Seminar mit Fokus auf unterschiedliche Erzähltechniken und dramaturgische Grundlagen (etwa Figurenentwicklung und Heldenreise). Entwicklung und Präsentation einer eigenen Geschichte.
}
\cventry{02/17--02/19}{Ehrenamt}{Kunstprojekt Web-Residency: x-temporary.org}{Berlin}{Webdesign}{
KünstlerInnen Residenz im digitalen Raum. Aufgabe waren die Überarbeitung der Webseite und die Unterstützung der KünstlerInnen in technischen Belangen.
}
\cventry{04/15--09/16}{Ehrenamt}{Studentenzeitung ad rem}{Dresden}{Journalist}{
Recherche und Erstellung journalistischer Artikel über Technik- und Digitalthemen, Hochschulpolitik sowie Veranstaltungen. Leitung für das Ressort Technik.
}
\cventry{08/15--04/16}{Ehrenamt}{Web-Applikation afeefa.de}{Dresden}{Webdesigner, Ruby-on-Rails-Entwickler}{
Mitarbeit an einer digitalen Plattform zu Angebot und Teilnahme für geflüchtete Menschen.
}

\newpage

\section{Sprachen}
\cvitemwithcomment{Deutsch}{Muttersprache}{}
\cvitemwithcomment{Englisch}{Gute Kenntnisse}{}
\cvitemwithcomment{Spanisch}{Grundkenntnisse}{}

\section{Technische Kenntnisse}

\cvline{Operating Systems}{Windows, macOS, Linux Ubuntu}

\cvline{Büro-Apps:}{openOffice/LibreOffice/Office, PowerPoint, \LaTeX}

\cvline{Programmier-sprachen: }{C\#, Java, C++, C}
\cvline{Scripting Sprachen: }{Python, R, Ruby}

\cvline{Datenbanken: }{SQL Server, MariaDB, PostreSQL, MongoDB}
\cvline{Game Engines: }{Unity, MonoGame}

\cvline{Web-Sprachen: }{HTML, CSS, JavaScript, PHP}
\cvline{Web Frameworks: }{Flask, Jinja, Bootstrap, React, Express.js}

\cvline{Web-CMS: }{Plone, Wordpress}
\cvline{Technologien:}{Git, Visual Studio Code, Apache, Nginx}

\cvline{Concept \& Collab Tools: } {Figma, Miro}

\cvline{Bild-bearbeitung: }{Gimp, Inkscape (Grundlegende Kenntnisse)}

\vspace{10pt}

Berlin, \today
~\\
~\\
%{\bfseries Waack}

\end{document}
%% end of file `template.tex'.
